%%%%%%%%%%%%%%%%%%%%%%%%%%%%%%%%%%%%%%%%%
% Medium Length Professional CV
% LaTeX Template
% Version 2.0 (8/5/13)
%
% This template has been downloaded from:
% http://www.LaTeXTemplates.com
%
% Original author:
% Rishi Shah 
%
% Current modifications by:
% Yashas ND
%
% Important note:
% This template requires the CV.cls file to be in the same directory as the
% .tex file. The CV.cls file provides the CV style used for structuring the
% document.
%
%%%%%%%%%%%%%%%%%%%%%%%%%%%%%%%%%%%%%%%%%

%----------------------------------------------------------------------------------------
%	PACKAGES AND OTHER DOCUMENT CONFIGURATIONS
%----------------------------------------------------------------------------------------

\documentclass{CV} % Use the custom CV.cls style
\usepackage[left=0.75in,top=0.6in,right=0.75in,bottom=0.6in]{geometry} % Document margins
\newcommand{\tab}[1]{\hspace{.2667\textwidth}\rlap{#1}}
\newcommand{\itab}[1]{\hspace{0em}\rlap{#1}}
\name{Yashas ND} % Your name
% \address{305 Dharam Palace, Gohaur Baug, Bilimora 396321} % Your address
% \address{123 Pleasant Lane \\ City, State 12345} % Your secondary addess (optional)
\address{+91 82771 17857 \\ \href{mailto:ndyashas@gmail.com}{ndyashas@gmail.com}} % Your phone number and email
\address{\href{https://ndyashas.github.io}{ndyashas.github.io}}

\begin{document}

%% \begin{rSection}{Objective}
%%   Actively seeking opportunities in Software Engineering and development.
%% \end{rSection}


%% \begin{rSection}{Interests}
%%   \begin{flushleft}
%%       Reinforcement Learning, Neuroscience, Swarm simulations, Hardware design, and Reconfigurable Computing.
%%   \end{flushleft}
%% \end{rSection}


%--------------------------------------------------------------------------------
%    Experiences
%-----------------------------------------------------------------------------------------------
\begin{rSection}{Experience}
  \begin{rSubsection}{Centre for Cloud Computing and Big Data, PES University}{August 2020 -- Present}{Junior Research Fellow}{Bengaluru, India}
    \begin{list}{$\cdot$}{\footnotesize}\itemsep -0.5em \vspace{-0.5em}
      \item Designing hardware accelerators on FPGAs.
    \end{list}
  \end{rSubsection}
  
  \begin{rSubsection}{CYRAN AI Solutions - Indian Institute of Technology, Delhi}{January 2020 -- June 2020}{Intern}{Delhi, India}
    \begin{list}{$\cdot$}{\footnotesize}\itemsep -0.5em \vspace{-0.5em}
      \item Worked on various applications powered by various ML architectures such as GANs, CNNs.
      \item Worked on software development for BUDDHI Kit, India’s first AI DIY Kit.
    \end{list}
  \end{rSubsection}

  \begin{rSubsection}{PES University}{August 2019 -- December 2019}{Teaching Assistant}{Bengaluru, India}
    \begin{list}{$\cdot$}{\footnotesize}\itemsep -0.5em \vspace{-0.5em}
      \item Worked as a TA for the course ``Digital Design and Computer Organization''.
      \item Assistance during the lab sessions.
    \end{list}
  \end{rSubsection}

  \begin{rSubsection}{NVM Research Lab - Indian Institute of Technology, Delhi}{May 2019 -- July 2019}{Summer Intern}{Delhi, India}
    \begin{list}{$\cdot$}{\footnotesize}\itemsep -0.5em \vspace{-0.5em}
      \item Investigated image related, and audio-related ML techniques using GANs, CNNs, and RNNs.
      \item Gained in-depth experience in ML frameworks such as TensorFlow and PyTorch.
    \end{list}
  \end{rSubsection}

  \begin{rSubsection}{Centre for Cloud computing and Big Data}{February 2018 -- December 2019}{Research Intern}{Bengaluru, India}
    \begin{list}{$\cdot$}{\footnotesize}\itemsep -0.5em \vspace{-0.5em}
      \item Hardware design for accelerated matrix multiplication on FPGAs.
      \item Utilization of elastic circuits techniques for handling critical path delays.
      \item Full flow from High-level logic description till final FPGA configuration and testing.
    \end{list}
  \end{rSubsection}

  \begin{rSubsection}{NextGen Earth Labs (NGEL)}{December 2018 -- May 2019}{Project Intern}{Bengaluru, India}
    \begin{list}{$\cdot$}{\footnotesize}\itemsep -0.5em \vspace{-0.5em}
      \item Developed cloud-based back-end to manage data logging and broadcasting control messages to nodes.
      \item Developed front end to visualize air quality data in different areas.
    \end{list}
  \end{rSubsection}

  \begin{rSubsection}{PACE PES}{August 2017 - August 2018}{Software development Intern}{Bengaluru, India}
    \begin{list}{$\cdot$}{\footnotesize}\itemsep -0.5em \vspace{-0.5em}
      \item Developed back-end for analyzing data procured from the electric vehicle.
      \item Development of software for vehicle control through multiple interfaces.
    \end{list}
  \end{rSubsection}
  
\end{rSection}



%----------------------------------------------------------------------------------------
%	EDUCATION SECTION
%----------------------------------------------------------------------------------------

\begin{rSection}{Education}


  \begin{rSubsection}{University of Southern California}{August 2021 - Present}{Graduate}{LA, CA, USA}
    \begin{list}{}{\footnotesize}\itemsep -0.5em \vspace{-0.5em}
      \item Majors in Computer Engineering. GPA : {\bf 4.0 / 4.0}
      \item - EE 457 Computer Systems Organization - Prof. Gandhi Puvvada
      \item - EE 477 MOS VLSI
    \end{list}
  \end{rSubsection}

  \begin{rSubsection}{PES University}{August 2016 - November 2020}{Undergraduate}{Bengaluru, India}
    \begin{list}{}{\footnotesize}\itemsep -0.5em \vspace{-0.5em}
      \item {\bf Specialization in \textit{Systems and Core Computing}.}
      \item Majors in Computer Science and Engineering. CGPA : {\bf 9.44 / 10}
      \item Minors in Electronics and Communication Engineering.
    \end{list}
  \end{rSubsection}
  
  \begin{rSubsection}{Manasarovar Pushkarini Vidyashram Pre-University College}{August 2014 - May 2016}{Pre-University study \tiny{(12th grade)}}{Mysuru, India}
    \begin{list}{}{\footnotesize}\itemsep -0.5em \vspace{-0.5em}
      \item Focus on Computer Science, Physics, Chemistry, and Mathematics. Score: {\bf $97$\%}.
    \end{list}
  \end{rSubsection}

  \begin{rSubsection}{Mahajana Public School \scriptsize{(Under CBSE of India)}}{August 2010 - May 2014}{10th-grade}{Mysuru, India}
    \begin{list}{}{\footnotesize}\itemsep -0.5em \vspace{-0.5em}
      \item Score: {\bf 10/10 CGPA}
    \end{list}
  \end{rSubsection}

\end{rSection}


%----------------------------------------------------------------------------------------
%	PATENTS
%----------------------------------------------------------------------------------------


\begin{rSection}{Patents}

  \begin{itemize}

  \item ``\textit{Elastic Pipelines for Data Processing. Application number 201941028170}'' filed at the Indian patent office. \
    \newline \textbf{Inventors:} Reetinder Sidhu, Vaibhav BV, \textbf{Yashas Nagavane Dattatreya}, Rachana Aithal KR.

  \item ``\textit{A system and method for real-time correction of collision-free flight paths of flying objects. \
    Application number 201941047692}'' filed at the Indian patent office.\newline \textbf{Inventors:} Antony Louis Piriyakumar Douglas, \
    \textbf{Yashas Nagavane Dattatreya}, Vivek Partal, Devashish Satyanarayan Vaishnav.
    
  \end{itemize}
  
\end{rSection}


%----------------------------------------------------------------------------------------
%	PAPERS AND PRESENTATIONS
%----------------------------------------------------------------------------------------


\begin{rSection}{Papers and Presentations}

  \begin{itemize}

  \item ``\textit{Hardware Accelerated Matrix Multiplication using a 400 MHz Systolic Array on a CPU+FPGA Platform.}'' by \
    Reetinder Sidhu, \textbf{Yashas ND}, Vishal Rao, Rachana Aithal, Vennela Katasani, Sneha Rao GR, and Vishal S. Poster \
    presentation at 33rd VLSI Design Conference 2020. Awarded the \textbf{best poster paper award in the user design \
    track category}.
    
  \end{itemize}
  
\end{rSection}



%----------------------------------------------------------------------------------------
%	PROJECTS
%----------------------------------------------------------------------------------------

\begin{rSection}{Projects}

  \begin{itemize}

  \item \textbf{\href{https://ndyashas.github.io/projects/Dhwani.html}{Dhwani}} \
    An \textit{English} to \textit{Indic} language phonetic conversion engine. 
    
  \item \textbf{Digital Design for Regular expression matching on FPGAs and implementation on PYNQ ZYNQ board:} \
    Designed circuit in Xilinx Vivado and tested using IPython framework on JupyterNotebook.

  \item \textbf{\href{https://github.com/ndyashas/Third-I-v2.0}{Third-I-v2.0:}} Implemented a user-space file system \
    on Linux using FUSE. Further, implemented extended functionalities such as soft-links and hard-links in operations.

  \item \textbf{Compiler for generating object code from parsing JavaScript code:} Built using ``lex'' and ``yacc'' tools, \
    implemented iterative backend optimizer for object code optimization.

  \item \textbf{Demonstrative CPU design:} Designed various parts such as ALU, Register File, Program Counter, and Control \
    Unit. Integrated all of these and tested for functional correctness of the design.
    
  \end{itemize}
  
\end{rSection}


%----------------------------------------------------------------------------------------
%	ACHIEVEMENTS
%----------------------------------------------------------------------------------------

\begin{rSection}{Achievements}

  \begin{itemize}

  \item Recipient of \textit{Prof. CNR Rao Merit Scholarship} from PES University from the past three years.

  \item Recipient of ``Distinction award'' for securing First class with Distinction in multiple semesters from PES University.
    
  \item \textbf{Ranked 290 out of 170,000} candidates in State level Engineering competitive Exam (KCET 2016).

  \item Completed \href{https://youtu.be/khK5RrpkqZM}{PACE Project} of \textit{Personal Urban Mobility Access}, \
    organized by General Motors, PACE Global Annual Forum, Warren, Michigan; July 2018.
    
  \end{itemize}
  
\end{rSection}


%% \begin{rSection}{Misc}

%% \begin{tabular}{ @{} >{\bfseries}l @{\hspace{6ex}} l }
%% Gender \ & Male\\
%% Date of Birth \ & 19th February 1999\\ 
%% Languages \ & English, Kannada, Hindi\\ \\
%% \end{tabular}

%% \end{rSection}
\end{document}
