%%%%%%%%%%%%%%%%%%%%%%%%%%%%%%%%%%%%%%%%%
% Medium Length Professional CV
% LaTeX Template
% Version 2.0 (8/5/13)
%
% This template has been downloaded from:
% http://www.LaTeXTemplates.com
%
% Original author:
% Rishi Shah 
%
% Current modifications by:
% Yashas ND
%
% Important note:
% This template requires the CV.cls file to be in the same directory as the
% .tex file. The CV.cls file provides the CV style used for structuring the
% document.
%
%%%%%%%%%%%%%%%%%%%%%%%%%%%%%%%%%%%%%%%%%

%----------------------------------------------------------------------------------------
%	PACKAGES AND OTHER DOCUMENT CONFIGURATIONS
%----------------------------------------------------------------------------------------

\documentclass{CV} % Use the custom CV.cls style
\usepackage[left=0.75in,top=0.6in,right=0.75in,bottom=0.6in]{geometry} % Document margins
\newcommand{\tab}[1]{\hspace{.2667\textwidth}\rlap{#1}}
\newcommand{\itab}[1]{\hspace{0em}\rlap{#1}}
\name{Yashas Nagavane Dattatreya} % Your name
% \address{305 Dharam Palace, Gohaur Baug, Bilimora 396321} % Your address
% \address{123 Pleasant Lane \\ City, State 12345} % Your secondary addess (optional)
\address{+1 323 847 3243 \\ %% +91 82771 17857 \\
  \href{mailto:ndyashas@gmail.com}{ndyashas@gmail.com}} % Your phone number and email
\address{\href{https://ndyashas.github.io}{ndyashas.github.io}}

\begin{document}

%% \begin{rSection}{Objective}
%%   Actively seeking opportunities in Software Engineering and development.
%% \end{rSection}


%% \begin{rSection}{Interests}
%%   \begin{flushleft}
%%       Reinforcement Learning, Neuroscience, Swarm simulations, Hardware design, and Reconfigurable Computing.
%%   \end{flushleft}
%% \end{rSection}


%--------------------------------------------------------------------------------
%    Experiences
%-----------------------------------------------------------------------------------------------
\begin{rSection}{Experience}
  \begin{rSubsection}{Centre for Cloud Computing and Big Data, PES University}{August 2020 -- June 2021}{RTL Design Engineer (Junior Research Fellow)}{Bengaluru, India}
    \begin{list}{$\cdot$}{\footnotesize}\itemsep -0.5em \vspace{-0.5em}
      Project 1: Hardware accelerated matrix multiplier
    \item Wrote RTL for a 64 × 64 cached matrix-multiplication accelerator on Intel Stratix 10 FPGA.
    \item Designed recursive high-speed full-cycle LFSRs for use in caching FIFOs.
    \item  Interfaced multiplier AFU with Intel CCI-P (Core Cache Interface).
      This achieves state-of-the-art throughput of 2.5 TFLOPS at 309MHz.


      Project 2: Cluster monitoring and controlling tool supporting
      \item Unified graphical view of cluster machines, User login activities, Remote reboot/shutdown support, RAM and CPU usage monitoring.
    \end{list}
  \end{rSubsection}

  \begin{rSubsection}{Centre for Cloud Computing and Big Data, PES University}{February 2018 -- August 2020}{Digital Logic Design Intern}{Bengaluru, India}
    \begin{list}{$\cdot$}{\footnotesize}\itemsep -0.5em \vspace{-0.5em}
    \item Wrote RTL for a 16 × 16 matrix-multiplication accelerator on Intel Arria 10, CPU+FPGA platform.
    \item Improved FMAX by 60\% from 250MHz to 400MHz over initial implementation Resulting in a peak of 204.8 GFLOPS.

      This was awarded the "Best Poster Presentation" award at the 2020 VLSID conference.
    \end{list}
  \end{rSubsection}
  
  \begin{rSubsection}{CYRAN AI Solutions - Indian Institute of Technology, Delhi}{January 2020 -- June 2020}{Intern}{Delhi, India}
    \begin{list}{$\cdot$}{\footnotesize}\itemsep -0.5em \vspace{-0.5em}
      \item Worked on various applications powered by various ML architectures such as GANs, CNNs.
      \item Worked on software development for BUDDHI Kit, India’s first AI DIY Kit.
    \end{list}
  \end{rSubsection}

  \begin{rSubsection}{PES University}{August 2019 -- December 2019}{Teaching Assistant}{Bengaluru, India}
    \begin{list}{$\cdot$}{\footnotesize}\itemsep -0.5em \vspace{-0.5em}
      \item Worked as a TA for the course ``Digital Design and Computer Organization''.
      \item Assistance during the lab sessions.
    \end{list}
  \end{rSubsection}

  \begin{rSubsection}{NVM Research Lab - Indian Institute of Technology, Delhi}{May 2019 -- July 2019}{Summer Intern}{Delhi, India}
    \begin{list}{$\cdot$}{\footnotesize}\itemsep -0.5em \vspace{-0.5em}
      \item Investigated image related, and audio-related ML techniques using GANs, CNNs, and RNNs.
      \item Gained in-depth experience in ML frameworks such as TensorFlow and PyTorch.
    \end{list}
  \end{rSubsection}

  %% \begin{rSubsection}{Centre for Cloud computing and Big Data}{February 2018 -- December 2019}{Research Intern}{Bengaluru, India}
  %%   \begin{list}{$\cdot$}{\footnotesize}\itemsep -0.5em \vspace{-0.5em}
  %%     \item Hardware design for accelerated matrix multiplication on FPGAs.
  %%     \item Utilization of elastic circuits techniques for handling critical path delays.
  %%     \item Full flow from High-level logic description till final FPGA configuration and testing.
  %%   \end{list}
  %% \end{rSubsection}

  %% \begin{rSubsection}{NextGen Earth Labs (NGEL)}{December 2018 -- May 2019}{Project Intern}{Bengaluru, India}
  %%   \begin{list}{$\cdot$}{\footnotesize}\itemsep -0.5em \vspace{-0.5em}
  %%     \item Developed cloud-based back-end to manage data logging and broadcasting control messages to nodes.
  %%     \item Developed front end to visualize air quality data in different areas.
  %%   \end{list}
  %% \end{rSubsection}

  \begin{rSubsection}{PACE PES}{August 2017 - August 2018}{Software development Intern}{Bengaluru, India}
    \begin{list}{$\cdot$}{\footnotesize}\itemsep -0.5em \vspace{-0.5em}
      \item Developed back-end for analyzing data procured from the electric vehicle.
      \item Development of software for vehicle control through multiple interfaces.
    \end{list}
  \end{rSubsection}
  
\end{rSection}



%----------------------------------------------------------------------------------------
%	EDUCATION SECTION
%----------------------------------------------------------------------------------------

\begin{rSection}{Education}


  \begin{rSubsection}{University of Southern California}{August 2021 - Present}{Graduate}{LA, CA, USA}
    \begin{list}{}{\footnotesize}\itemsep -0.5em \vspace{-0.5em}
      \item Majors in Computer Engineering. GPA : {\bf 4.0 / 4.0}
      \item - EE 557  Computer Systems Architecture
      \item - EE 577a VLSI System Design - 1
      \item - EE 457  Computer Systems Organization - Prof. Gandhi Puvvada
      \item - EE 477  MOS VLSI
    \end{list}
  \end{rSubsection}

  \begin{rSubsection}{PES University}{August 2016 - November 2020}{Undergraduate}{Bengaluru, India}
    \begin{list}{}{\footnotesize}\itemsep -0.5em \vspace{-0.5em}
      \item {\bf Specialization in \textit{Systems and Core Computing}.}
      \item Majors in Computer Science and Engineering. CGPA : {\bf 9.44 / 10}
      \item Minors in Electronics and Communication Engineering.
    \end{list}
  \end{rSubsection}
  
  \begin{rSubsection}{Manasarovar Pushkarini Vidyashram Pre-University College}{August 2014 - May 2016}{Pre-University study \tiny{(12th grade)}}{Mysuru, India}
    \begin{list}{}{\footnotesize}\itemsep -0.5em \vspace{-0.5em}
      \item Focus on Computer Science, Physics, Chemistry, and Mathematics. Score: {\bf $97$\%}.
    \end{list}
  \end{rSubsection}

  \begin{rSubsection}{Mahajana Public School \scriptsize{(Under CBSE of India)}}{August 2010 - May 2014}{10th-grade}{Mysuru, India}
    \begin{list}{}{\footnotesize}\itemsep -0.5em \vspace{-0.5em}
      \item Score: {\bf 10/10 CGPA}
    \end{list}
  \end{rSubsection}

\end{rSection}


%----------------------------------------------------------------------------------------
%	PATENTS
%----------------------------------------------------------------------------------------


\begin{rSection}{Patents}

  \begin{itemize}

  \item ``\textit{Elastic Pipelines for Data Processing. Application number 201941028170}'' filed at the Indian patent office. \
    \newline \textbf{Inventors:} Reetinder Sidhu, Vaibhav BV, \textbf{Yashas Nagavane Dattatreya}, Rachana Aithal KR.

  \item ``\textit{A system and method for real-time correction of collision-free flight paths of flying objects. \
    Application number 201941047692}'' filed at the Indian patent office.\newline \textbf{Inventors:} Antony Louis Piriyakumar Douglas, \
    \textbf{Yashas Nagavane Dattatreya}, Vivek Partal, Devashish Satyanarayan Vaishnav.
    
  \end{itemize}
  
\end{rSection}


%----------------------------------------------------------------------------------------
%	PAPERS AND PRESENTATIONS
%----------------------------------------------------------------------------------------


\begin{rSection}{Papers and Presentations}

  \begin{itemize}

  \item ``\textit{Hardware Accelerated Matrix Multiplication using a 400 MHz Systolic Array on a CPU+FPGA Platform.}'' by \
    Reetinder Sidhu, \textbf{Yashas ND}, Vishal Rao, Rachana Aithal, Vennela Katasani, Sneha Rao GR, and Vishal S. Poster \
    presentation at 33rd VLSI Design Conference 2020. Awarded the \textbf{best poster paper award in the user design \
    track category}.
    
  \end{itemize}
  
\end{rSection}



%----------------------------------------------------------------------------------------
%	PROJECTS
%----------------------------------------------------------------------------------------

\begin{rSection}{Projects}

  %% \begin{itemize}

  %% \item \textbf{\href{https://ndyashas.github.io/projects/Dhwani.html}{Dhwani}} \
  %%   An \textit{English} to \textit{Indic} language phonetic conversion engine.


  %% \item \textbf{} \


  %% \end{itemize}


  \begin{rSubsection}{\href{https://ndyashas.github.io/projects/Salaga.html}{RISC-V cores supporting rv32i instruction set}}{}{System Verilog | C++ | Verilator | GTKWave}{}
    \begin{list}{}{\footnotesize}\itemsep -0.5em \vspace{-0.5em}
    \item Designing RISC-V cores supporting the rv32i instruction set.
    \item Simulation of supporting modules such as L1 Instruction and data cache, and stalls.
    \item Supporting modules written in C++ and simulated using Verilator, and debugged using GTKWave.
    \end{list}
  \end{rSubsection}

  \begin{rSubsection}{5-stage pipelined MIPS Processor Design}{}{Verilog | ModelSim}{}
    \begin{list}{}{\footnotesize}\itemsep -0.5em \vspace{-0.5em}
    \item Designed a pipelined 5-stage CPU with an internally forwarding register file to execute MIPS R-type, Branch, and mem instructions using RTL (Register Transfer Level) coding in Verilog and simulating on ModelSim.
    \item Design included forwarding and hazard detection unit to prevent the Read-After-Write data dependencies by stalling for the early and late branch designs.
    \end{list}
  \end{rSubsection}

  \begin{rSubsection}{Digital Spiking Neuron Design}{}{Cadence Virtuoso}{}
    \begin{list}{}{\footnotesize}\itemsep -0.5em \vspace{-0.5em}
    \item Designed and implemented schematic and layout of a digital circuit that mimics a spiking neuron in with Cadence Virtuoso.
    \item Simulated the digital spiking neuron on SPECTRE and performed the timing analysis of the signals. Validated the physical design with DRC and LVS and optimized it to have a minimum area-delay product.
    \end{list}
  \end{rSubsection}

\newpage
  
  \begin{rSubsection}{\href{https://github.com/SofDevs-Do/R-LMON}{R-LMON Systems Monitoring Tool for cluster systems}}{}{Bash | Python | JavaScript}{}
    \begin{list}{}{\footnotesize}\itemsep -0.5em \vspace{-0.5em}
      \item Cluster monitoring and controlling tool. Used at Center for Cloud Computing and Big Data, PES University, and now at AMD Bangalore.
      \item Unified graphical view of cluster machines, User login activities, Remote reboot/shutdown support, RAM and CPU usage monitoring.
    \end{list}
  \end{rSubsection}

  \begin{rSubsection}{High-speed full-cycle recursive Linear-Feedback-Shift-Registers}{}{iverilog | GTKWave}{}
    \begin{list}{}{\footnotesize}\itemsep -0.5em \vspace{-0.5em}
    \item Wrote recursive HLS design for a full-cycle / arbitrary-cycle Linear Feedback Shift Register.
    \item Test scripts in Python for functional verification.
    \item Written in Flo-Hask HLS framework in Haskell, developed by Dr. Reetinder Sidhu.
    \item Simulation and waveform viewing done was done using iverilog, and GTKWave.
    \end{list}
  \end{rSubsection}

  \begin{rSubsection}{Regular expression matching on FPGA}{}{Vivado | Verilog | iverilog | GTKWave}{}
    \begin{list}{}{\footnotesize}\itemsep -0.5em \vspace{-0.5em}
    \item Non-deterministic Finite Automata-based regular expression matching.
    \item Tested on PYNQ ZYNQ FPGA board.
    \item Written in Flo-Hask HLS framework in Haskell, developed by Dr. Reetinder Sidhu.
    \end{list}
  \end{rSubsection}

  \begin{rSubsection}{\textbf{\href{https://github.com/ndyashas/Third-I-v2.0}{Third-I-v2.0:}}}{}{C | FUSE}{}
    \begin{list}{}{\footnotesize}\itemsep -0.5em \vspace{-0.5em}
    \item Implemented a user-space file system on Linux using FUSE.
    \item Implemented extended functionalities such as soft-links and hard-links in operations.
    \end{list}
  \end{rSubsection}

  %% \item \textbf{\href{https://ndyashas.github.io/projects/Dhwani.html}{Dhwani}} \
  %%   An \textit{English} to \textit{Indic} language phonetic conversion engine.

  \begin{rSubsection}{\textbf{\href{https://ndyashas.github.io/projects/Dhwani.html}{Dhwani}}}{}{Python}{}
    \begin{list}{}{\footnotesize}\itemsep -0.5em \vspace{-0.5em}
    \item An \textit{English} to \textit{Indic} language phonetic conversion tool.
    \end{list}
  \end{rSubsection}

\end{rSection}


%----------------------------------------------------------------------------------------
%	ACHIEVEMENTS
%----------------------------------------------------------------------------------------

\begin{rSection}{Achievements}

  \begin{itemize}

  \item \textbf{MS Honors Program} admitted to the MS Honors program at USC where < 5\% of students are admitted for maintaining a 4.0 GPA.

  \item Recipient of \textit{Prof. CNR Rao Merit Scholarship} from PES University from the past three years.

  \item Recipient of ``Distinction award'' for securing First class with Distinction in multiple semesters from PES University.
    
  \item \textbf{Ranked 290 out of 170,000} candidates in State level Engineering competitive Exam (KCET 2016).

  \item Completed \href{https://youtu.be/khK5RrpkqZM}{PACE Project} of \textit{Personal Urban Mobility Access}, \
    organized by General Motors, PACE Global Annual Forum, Warren, Michigan; July 2018.
    
  \end{itemize}
  
\end{rSection}


%% \begin{rSection}{Misc}

%% \begin{tabular}{ @{} >{\bfseries}l @{\hspace{6ex}} l }
%% Gender \ & Male\\
%% Date of Birth \ & 19th February 1999\\ 
%% Languages \ & English, Kannada, Hindi\\ \\
%% \end{tabular}

%% \end{rSection}
\end{document}
